\documentclass{article}
\usepackage[utf8]{inputenc}
\usepackage{apacite}
\usepackage{graphicx}   
\graphicspath{{figures/}}

\title{Spatiotemporal Pointing in Augmented Reality Music Teaching Systems}
\author{Jordan Aiko Deja}

\begin{document}
%\maketitle
\begin{center}
\large \textbf{Spatiotemporal Pointing in Augmented Reality Music Teaching Systems}
\\            
\vspace{0.5cm}\\
A Research Proposal\\
\vspace{0.5cm}
Jordan Aiko Deja\\
jordan.deja@famnit.upr.si\\
\vspace{0.5cm}
March 23 2020
\vspace{0.5cm}
\end{center}
\nocite{*}

\begin{abstract}
     Temporal pointing allows the user to touch base with a moving target within a limited window time. Movement and control is limited which then gives the user a short window to determine when to launch a valid sensory response. This phenomenon can be observed in augmented visualizations especially those that require temporal precision, rhythm and synchronization. Similarly, error rates are as well difficult to track which could lead to design and performance issues in these systems. Music teaching systems are a special group of systems that qualify to have the same qualities where temporal pointing can be best investigates. This research attempts to investigate, apply and validate the spatiotemporal models for error tracking in music systems that use augmented reality and similar visualizations. 
\end{abstract}


\section{Introduction}
\section{Previous Research}
\section{Research Plan and Methodology}
\subsection{Research Questions}
research question\\
what are the motives to classify information as important and unimportant? are there any differences in classifying information as important across collections? is information classified as important managed differently? \\
hypothesis statement\\
information with added value (created or found) and on which we spand more time is more important and we take more care of it (categorizing, filing, etc)\\
supporting information\\
\subsection{Methodology}
discuss the different main phases of the study\\
what information do we need to collect?\\
describe this information and their types and why they are important\\
what are the artefacts and deliverables at the end of each phase?\\
synthesis of all the main phases and how they lead to the contribution or to answering the research question\\
\section{Contribution}
have a brief summary of what the most recent contributions there are and identify how your contribution stands out. 
\bibliographystyle{apacite}
\bibliography{references}

\end{document}
