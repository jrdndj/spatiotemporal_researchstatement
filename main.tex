\documentclass{article}
\usepackage[utf8]{inputenc}
\usepackage{apacite}
\usepackage{graphicx}   
\graphicspath{{figures/}}

\title{Spatiotemporal Pointing in Augmented Reality Music Teaching Systems}
\author{Jordan Aiko Deja}

\begin{document}
%\maketitle
\begin{center}
\large \textbf{Spatiotemporal Pointing in Augmented Reality Music Teaching Systems}
\\            
\vspace{0.5cm}\\
A Research Proposal\\
\vspace{0.5cm}
Jordan Aiko Deja\\
jordan.deja@famnit.upr.si\\
\vspace{0.5cm}
March 23 2020
\vspace{0.5cm}
\end{center}

\begin{abstract}
     Spatiotemporal pointing allows the user to touch base with a moving target within a limited time frame especially in a virtual space. In this given short window, the user has limited time to move, control its movement and even process the type of corresponding response that they should render. This phenomenon can be observed in any environment augmented by visual or audio cues - environments that require timely precision, rhythm and synchronization. Similarly, error rates are commonly observed but are not properly modelled in order to mitigate them. Movements and precision are recognized differently by the computer as there are other considerations beyond flat surfaces that are usually covered by Fitts' Law. This research attempts to understand spatiotemporal pointing analysis and modeling in augmented reality music teaching systems. Future work can lead to the use of these findings towards better design affordances in music teaching systems. 
\end{abstract}


\section{Introduction}
Temporal pointing refers to having a target that is about to appear within a limited time window for the user to response \cite{lee2016website}. Spatial pointing on the other hand requires the user to have control of the movement in an immersive space. As such Spatiotemporal pointing combines these two principles and is referred to as the movement of a user in an immersive environment as a response to a stimulus that is driven by a limited amount of time for selection. In such environments, there are many considerations that need to be understood. To maintain the temporal element, an immersive space needs to have an internal time-keeping mechanism, a response-execution stage, input processing done automatically all placed in a virtual non 2-dimensional space. There is a lot of work being done in this area with the aim of investigating and understanding human cognitive load, improving the accuracy in the rendering and performance of visual elements and discovering novel implications that minimise latency.\\

Innovations in spatiotemporal pointing contributes to Computational HCI. Since immersive environments are computer-generated, it is indeed vital to automate and streamline the processes that improve the modelling of the activities that happen in these environments such as spatiotemporal pointing. 
\section{Related Work}
The study of \cite{lee2017boxer}, investigated on multimodal collision of virtual objects. They compared temporal pointing vs virtual batting in terms of spatial precision in collisions. This was on a virtual ball game and not on a music teaching system. 
\section{Research Plan and Methodology}
\subsection{Research Questions}
research question\\
what are the motives to classify information as important and unimportant? are there any differences in classifying information as important across collections? is information classified as important managed differently? \\
hypothesis statement\\
information with added value (created or found) and on which we spand more time is more important and we take more care of it (categorizing, filing, etc)\\
supporting information\\
\subsection{Methodology}
discuss the different main phases of the study\\
what information do we need to collect?\\
describe this information and their types and why they are important\\
what are the artefacts and deliverables at the end of each phase?\\
synthesis of all the main phases and how they lead to the contribution or to answering the research question\\
\section{Contribution}
have a brief summary of what the most recent contributions there are and identify how your contribution stands out. 
\bibliographystyle{apacite}
\bibliography{references}

\end{document}
